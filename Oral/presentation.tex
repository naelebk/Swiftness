\documentclass{beamer}

\usepackage[utf8x]{inputenc}
\usepackage{amsfonts,amsmath,oldgerm}
\usepackage[T1]{fontenc}
%\usepackage{lmodern} 
\usepackage{amsmath}
\usepackage{graphicx}
\usepackage{amsfonts}
\usepackage{amssymb}
\usepackage{tcolorbox}
\usepackage{xcolor}
\usepackage{multicol}
\usepackage[french]{babel}
\usepackage{etoolbox, tikz, geometry}
\usepackage{hyperref}
\usepackage{xcolor} %Package pour les couleurs
\usepackage{epsfig} %Package pour insérer des images faites avec Xfig
\usepackage{fancyhdr} %Package pour créer des en-têtes et pieds de page
\usepackage{listings}
\lstset{
language=C,                  % choose the language of the code
basicstyle=\footnotesize,       % the size of the fonts that are used for the code
numbers=left,                   % where to put the line-numbers
numberstyle=\footnotesize,      % the size of the fonts that are used for the line-numbers
stepnumber=2,                   % the step between two line-numbers. If it's 1 each line will be numbered
numbersep=5pt,                  % how far the line-numbers are from the code
backgroundcolor=\color{white},  % choose the background color. You must add \usepackage{color}
showspaces=false,               % show spaces adding particular underscores
showstringspaces=false,         % underline spaces within strings
showtabs=false,                 % show tabs within strings adding particular underscores
frame=single,	                % adds a frame around the code
tabsize=2	                % sets default tabsize to 2 spaces
}
\usepackage{fancyhdr}
\date{}
% -------------------------- THEME ----------------------------------
\usetheme{Singapore}
\usecolortheme{orchid}
\useinnertheme[shadow]{rounded}
%\useoutertheme{infolines}
\newcommand{\testcolor}[1]{\colorbox{#1}{\textcolor{#1}{test}}~\texttt{#1}}
\newcommand{\hrefcol}[2]{\textcolor{cyan}{\href{#1}{#2}}}


% -------------------------- THEME ----------------------------------


\DeclareMathOperator {\argmin}{argmin}
\title{Swiftness}
\subtitle{Soutenance de projet de Licence \\ \medskip Licence 3 Informatique, 2023-2024 \\ \medskip (jour de semaine) 27/28/29 Mars 2024}
\author{EMBARKI Naël - GAUTHIER Julien - HUMBERT Théo}
\AtBeginSubsection{%
  \begin{frame}
  	\frametitle{Plan}
  	\tableofcontents[currentsection, currentsubsection]
  \end{frame}
}

\AtBeginSection{%
  \begin{frame}
  	\frametitle{Plan}
  	\tableofcontents[currentsection]
  \end{frame}
}

\begin{document}
\setbeamertemplate{navigation symbols}{} 
\begin{frame}
\thispagestyle{empty}
\begin{figure}
\begin{flushleft}
\begin{minipage}{10cm}
\hspace*{-1cm}
\vspace*{-14cm}
\includegraphics[width=0.25\linewidth]{logo_st.png}
\end{minipage}
\end{flushleft}
\end{figure}

\titlepage
\end{frame}

\setbeamertemplate{navigation symbols}{\tiny {\insertframenumber /\inserttotalframenumber}}
\begin{frame}{Plan}
\tableofcontents
\end{frame}
\section{Introduction}
\begin{frame}
\frametitle{Remerciements}
\begin{tcolorbox}[colback=yellow!20!white,colframe=yellow!60!black]
Avant toute chose, nous tenons à sincèrement remercier \textbf{M. BERNARD}, enseignant référent de ce projet, pour : 
\begin{itemize}
\item suivi très régulier du projet ;
\item conseils, \textit{trucs et astuces}, techniques de développement (\textit{de jeux vidéos});
\item accessibilité, réactivité
\item ... 
\end{itemize}
\end{tcolorbox}
\end{frame}
\begin{frame}
\frametitle{Organisation de la soutenance}
\framesubtitle{Découpage}
Dans l'absolu, cette soutenance sera découpée en 3 grandes parties :
\begin{itemize}
\item \textit{Preludio} (conception - technologies utilisées - systèmes utilisés)
\item \textit{Korpo} (développement - techniques - physique du jeu)
\item \textit{Postludio} (résultats - conclusion - bilan)
\end{itemize}
\end{frame}
\section{Analyse - Conception}
\subsection{Technologies (langages - API)}
\begin{frame}
\frametitle{Langages de programmation utilisés}
\begin{itemize}
\item C++ (majoritairement) $\Rightarrow$ le jeu
\item Shell $\Rightarrow$ lancement du jeu (compilations et liaisons automatisées par scripts)
\item à repréciser (peut-être C, XML à voir par la suite)
\end{itemize}
\end{frame}
\begin{frame}
\frametitle{API utilisées}
\begin{itemize}
\item C++ $\Rightarrow$ \textbf{GF} (\textit{Gamedev Framework})
\end{itemize}
\end{frame}
\subsection{Modélisation}
\begin{frame}
\frametitle{Use case}
\end{frame}
\begin{frame}
\frametitle{Classes}
\end{frame}
\section{Développement}
\section{Résultats}
\section{Bilan}
\begin{frame}
\frametitle{The end !}
\framesubtitle{\textit{C'est tout pour le moment...}}
\begin{tcolorbox}[colback=purple!20!white,colframe=purple!60!black]
\begin{center}
Merci pour votre attention ! Des questions ?
\end{center}
\end{tcolorbox}
\end{frame}
\end{document}
